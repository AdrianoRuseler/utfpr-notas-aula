
% ---
\chapter{Relações Trigonométricas}
% ---

\section{Deslocamentos Angulares}
\subsection{Deslocamento de 90 graus}
\begin{eqnarray}
\sin(\theta + \tfrac{\pi}{2}) &= +\cos \theta \\
\cos(\theta + \tfrac{\pi}{2}) &= -\sin \theta \\
\tan(\theta + \tfrac{\pi}{2}) &= -\cot \theta \\
\csc(\theta + \tfrac{\pi}{2}) &= +\sec \theta \\
\sec(\theta + \tfrac{\pi}{2}) &= -\csc \theta \\
\cot(\theta + \tfrac{\pi}{2}) &= -\tan \theta
\end{eqnarray}

\subsection{Deslocamento de 180 graus}
\begin{eqnarray}
\sin(\theta + \pi) &= -\sin \theta \\
\cos(\theta + \pi) &= -\cos \theta \\
\tan(\theta + \pi) &= +\tan \theta \\
\csc(\theta + \pi) &= -\csc \theta \\
\sec(\theta + \pi) &= -\sec \theta \\
\cot(\theta + \pi) &= +\cot \theta 
\end{eqnarray}

\subsection{Deslocamento de 360 graus}
\begin{eqnarray}
\sin(\theta + 2\pi) &= +\sin \theta \\
\cos(\theta + 2\pi) &= +\cos \theta \\
\tan(\theta + 2\pi) &= +\tan \theta \\
\csc(\theta + 2\pi) &= +\csc \theta \\
\sec(\theta + 2\pi) &= +\sec \theta \\
\cot(\theta + 2\pi) &= +\cot \theta
\end{eqnarray}




\section{Relações de soma e subtração}

\begin{eqnarray}
\sin(\alpha \pm \beta) & =& \sin \alpha \cos \beta \pm \cos \alpha \sin \beta \\
\cos(\alpha \pm \beta) & =& \cos \alpha \cos \beta \mp \sin \alpha \sin \beta \\
\tan(\alpha \pm \beta) &= &\frac{\tan \alpha \pm \tan \beta}{1 \mp \tan \alpha \tan \beta}\\
\arcsin\alpha \pm \arcsin\beta &=& \arcsin\left(\alpha\sqrt{1-\beta^2} \pm \beta\sqrt{1-\alpha^2}\right)\\
\arccos\alpha \pm \arccos\beta &=& \arccos\left(\alpha\beta \mp \sqrt{(1-\alpha^2)(1-\beta^2)}\right)\\
\arctan\alpha \pm \arctan\beta &=&\arctan\left(\frac{\alpha \pm \beta}{1 \mp \alpha\beta}\right)
\end{eqnarray}



\section{Ângulo duplo}


\begin{eqnarray}
\sin 2\theta &=& 2 \sin \theta \cos \theta  = \frac{2 \tan \theta} {1 + \tan^2 \theta} \\
\cos 2\theta &=& \cos^2 \theta - \sin^2 \theta = 2 \cos^2 \theta - 1 = 1 - 2 \sin^2 \theta = \frac{1 - \tan^2 \theta} {1 + \tan^2 \theta}\\
\tan 2\theta &=& \frac{2 \tan \theta} {1 - \tan^2 \theta}\\
\cot 2\theta &=& \frac{\cot^2 \theta - 1}{2 \cot \theta}
\end{eqnarray}


\section{Ângulo Triplo}


\begin{eqnarray}
\sin 3\theta &=& - \sin^3\theta + 3 \cos^2\theta \sin\theta 
= - 4\sin^3\theta + 3\sin\theta  \\
\cos 3\theta  &=& \cos^3\theta - 3 \sin^2 \theta\cos \theta =
4 \cos^3\theta - 3 \cos\theta \\
\tan 3\theta &=& \frac{3 \tan\theta - \tan^3\theta}{1 - 3 \tan^2\theta}\\
\cot 3\theta &=& \frac{3 \cot\theta - \cot^3\theta}{1 - 3 \cot^2\theta}
\end{eqnarray}


\section{Meio ângulo}

\begin{eqnarray}
\sin \frac{\theta}{2} &=& \sgn \left(2 \pi - \theta + 4 \pi \left\lfloor \frac{\theta}{4\pi} \right\rfloor \right) \sqrt{\frac{1 \! - \! \cos \theta}{2}}\\
\sin^2\frac{\theta}{2} &=& \frac{1-\cos\theta}{2}\\
\cos \frac{\theta}{2} &=& \sgn \left(\pi + \theta + 4 \pi \left\lfloor \frac{\pi - \theta}{4\pi} \right\rfloor \right) \sqrt{\frac{1 + \cos\theta}{2}}\\
\cos^2\frac{\theta}{2} &=&\frac{1+\cos\theta}{2}\\
\tan \frac{\theta}{2} &=& \csc \theta - \cot \theta = \pm\, \sqrt{1 - \cos \theta \over 1 + \cos \theta} = \frac{\sin \theta}{1 + \cos \theta} = \frac{1-\cos \theta}{\sin \theta} \\
\tan\frac{\eta+\theta}{2} &=& \frac{\sin\eta+\sin\theta}{\cos\eta+\cos\theta} \\
\tan\left(\frac{\theta}{2} + \frac{\pi}{4}\right) &= &\sec\theta + \tan\theta \\
\sqrt{\frac{1 - \sin\theta}{1 + \sin\theta}}  &= &\frac{1 - \tan(\theta/2)}{1 + \tan(\theta/2)}\\
\tan\tfrac{1}{2}\theta  &=& \frac{\tan\theta}{1 + \sqrt{1+\tan^2\theta}} \left(-\tfrac{\pi}{2} < \theta < \tfrac{\pi}{2} \right)\\
\cot \frac{\theta}{2} & =& \csc \theta + \cot \theta = \pm\, \sqrt{1 + \cos \theta \over 1 - \cos \theta} = \frac{\sin \theta}{1 - \cos \theta} = \frac{1 + \cos \theta}{\sin \theta} 
\end{eqnarray}


\section{Redução de Potência}

\begin{eqnarray}
\sin^2\theta &=& \frac{1 - \cos 2\theta}{2}\\
\cos^2\theta &=& \frac{1 + \cos 2\theta}{8}\\
\sin^2\theta \cos^2\theta &=& \frac{1 - \cos 4\theta}{8}
\end{eqnarray}


\begin{eqnarray}
\sin^3\theta &=& \frac{3 \sin\theta - \sin 3\theta}{4}\\
\cos^3\theta &=& \frac{3 \cos\theta + \cos 3\theta}{4}\\
\sin^3\theta \cos^3\theta &=& \frac{3\sin 2\theta - \sin 6\theta}{32}
\end{eqnarray}



\section{Produto para soma}
\begin{eqnarray}
2\cos \theta \cos \varphi &=& {{\cos(\theta - \varphi) + \cos(\theta + \varphi)}}\\
2\sin \theta \sin \varphi &=& {{\cos(\theta - \varphi) - \cos(\theta + \varphi)} }\\
2\sin \theta \cos \varphi &=& {{\sin(\theta + \varphi) + \sin(\theta - \varphi)} }\\
2\cos \theta \sin \varphi &=& {{\sin(\theta + \varphi) - \sin(\theta - \varphi)} }\\
\tan \theta \tan \varphi &=&\frac{\cos(\theta-\varphi)-\cos(\theta+\varphi)}{\cos(\theta-\varphi)+\cos(\theta+\varphi)}
\end{eqnarray}


\section{Soma para Produto}
\begin{eqnarray}
\sin \theta \pm \sin \varphi &=& 2 \sin\left( \frac{\theta \pm \varphi}{2} \right) \cos\left( \frac{\theta \mp \varphi}{2} \right)\\
\cos \theta + \cos \varphi &=& 2 \cos\left( \frac{\theta + \varphi} {2} \right) \cos\left( \frac{\theta - \varphi}{2} \right)\\
\cos \theta - \cos \varphi &=& -2\sin\left( {\theta + \varphi \over 2}\right) \sin\left({\theta - \varphi \over 2}\right)\\
\end{eqnarray}

